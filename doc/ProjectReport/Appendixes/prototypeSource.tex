\chapter{Prototype source code}
\label{app:prototypeCode}

 \section{Main executable}
  The main executable is communicating with the OpenBeacon node and will retrieve user commands through a message queue.
  \singlespacing
  \begin{lstlisting}[
  	showstringspaces=false,
  	breaklines,
  	basicstyle=\footnotesize,
  	numbers=left,
  	language=C++,
  	frame=single,label=lst:prototypeMain,captionpos=b,caption={main.cpp of prototype}]
#include <fcntl.h>
#include <termios.h>
#include <stdio.h>
#include <string.h>
#include <unistd.h>
#include "msgq.h"

#define BAUDRATE B115200
#define OPENBEACONMODEMDEVICE "/dev/ttyACM0"

int main(int argc, char ** argv)
{
  bool run = true;
  int fd;
  int res;
  struct termios newtio;
  char buf[255];
  int readId;
  struct message  msg;
  int result;
  char command[255];

  // open device in rw mode, and as no controlling tty (is better
  // because if the device sends CTRL+C it will not affect the current
  // running console)
  fd = open ( OPENBEACONMODEMDEVICE, O_RDWR | O_NOCTTY | O_NONBLOCK );
  if (fd < 0)
  {
    perror (OPENBEACONMODEMDEVICE);
    return (-1);
  }

  // create message queue
  if ( (readId = msgget (COMMUNICATIONKEY, PERMS | IPC_CREAT)) < 0)
  {
    // print error
    return (-1);
  }

  // clear struct
  bzero(&newtio, sizeof(newtio));
  // setting control modes -> man 3 termios
  newtio.c_cflag = BAUDRATE | CRTSCTS | CS8 | CLOCAL | CREAD;
  // setting input modes -> man 3 termios
  newtio.c_iflag = IGNPAR | ICRNL;
  // setting output modes -> man 3 termios
  newtio.c_oflag = 0;
  // canonical input
  newtio.c_lflag = ICANON;
  // initialize all control characters
  newtio.c_cc[VTIME] = 0;  // inter-character timer unused
  newtio.c_cc[VMIN] = 2;  // blocking read until 1 character arrives
  // clean the line and set the new settings
  tcflush(fd, TCIFLUSH);
  tcsetattr(fd,TCSANOW,&newtio);

  while (run)
  {
    // read from OpenBeacon
    res = read(fd,buf,255);
    buf[res]='\0';
    if (buf[0] != '\n')
    {
      printf("%s", buf);
      buf[0] = '\0';
    }

    // read from msgq
    result = msgrcv(readId, (void *) &msg, sizeof(msg.text), 0, MSG_NOERROR | IPC_NOWAIT);
    if (result > 0)
    {
      printf ("Got msgq command: %s\n", msg.text);
      if (strncmp (msg.text, "quit", 4) == 0)
      {
        run = false;
      }
      else
      {
        strcpy (command, msg.text);
        strcat (command, "\r");
        write (fd, command, 2);
      }
    }
  }
  // delete message queue
  if (msgctl (readId, IPC_RMID, (struct msqid_ds *) 0) < 0)
  {
    // print error
    return (-1);
  }
  close (fd);
  return (0);
}

  \end{lstlisting}
  \onehalfspacing

 \section{Send executable}
  This executable is used to send commands to the main executable. Those commands are sent through a message queue
  \singlespacing
  \begin{lstlisting}[
  	showstringspaces=false,
  	breaklines,
  	basicstyle=\footnotesize,
  	numbers=left,
  	language=C,
  	frame=single,label=lst:prototypeSend,captionpos=b,caption={send-msg.c of prototype}]
#include <stdio.h>
#include <string.h>
#include "msgq.h"

int main (int argc, char ** argv)
{
  struct message  msg;
  int		  writeId;

  if (argc != 2)
  {
    printf ("wrong count of arguments\n");
    return (1);
  }
  if ( (writeId = msgget (COMMUNICATIONKEY, 0)) < 0)
  {
    return (-1);
  }

  // send command to msgq
  msg.type = 1;
  strcpy(msg.text, argv[1]);
  if (msgsnd(writeId, (void *) &msg, sizeof(msg.text), IPC_NOWAIT) < 0)
  {
    printf ("error while sending\n");
    return (-1);
  }

  // delete message queue
  if (msgctl (writeId, IPC_RMID, (struct msqid_ds *) 0) < 0)
  {
    return (-1);
  }
  return (0);
}
  \end{lstlisting}
  \onehalfspacing
 \section{Include for common information}
  This include file holds the informations both sources have in common.
  \singlespacing
  \begin{lstlisting}[
  	showstringspaces=false,
  	breaklines,
  	basicstyle=\footnotesize,
  	numbers=left,
  	language=C,
  	frame=single,label=lst:prototypeInclude,captionpos=b,caption={mesq.h of prototype}]
#include <sys/msg.h>
#define COMMUNICATIONKEY 4711L
#define PERMS 0666

struct message
{
  long type;
  char text[5];
};
  \end{lstlisting}
  \onehalfspacing
 \section{CMakeLists.txt build file}
  This section shows how simple CMakeLists.txt file, which is needed to build a project with CMake, will be.
  \singlespacing
  \begin{lstlisting}[
  	showstringspaces=false,
  	breaklines,
  	basicstyle=\footnotesize,
  	numbers=left,
  	frame=single,label=lst:prototypeCMakeLists,captionpos=b,caption={CMakeLists.txt of prototype}]
PROJECT ("Openbeacon Prototype")
ADD_DEFINITIONS (-Wall)

SET (MAINBINARY communicateWithOB)
SET (SENDBINARY commandToCommunicate)

SET (MAINSRC
  main.cpp
)

SET (SENDSRC
  send-msg.c
)

ADD_EXECUTABLE (${MAINBINARY} ${MAINSRC})
ADD_EXECUTABLE (${SENDBINARY} ${SENDSRC})
  \end{lstlisting}
  \onehalfspacing
