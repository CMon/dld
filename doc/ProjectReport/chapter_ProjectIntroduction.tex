\chapter{Introduction}

\section{Background}
 Person localisation is a big issue nowadays, but mostly localisation is used to track down people who would not like to be tracked down. Most of the people think that location detection is only used in negative ways. This might be right in some ways but there are also cases where this detection might be used in everyday life. One of these cases is an audio stream that will follow the person or a stream that changes if a person moves from one place to another. An example of this can be used in museums where people have their own guided tour. They lend a wireless headset and then go from one exhibit to another and every time they reach the area around the exhibit, a voice will tell them what they want to know about it. Another example of use would be a sound stream that will follow the person from one room to another, keeping the noise level in rooms that are unallocated, as low as possible.

\section{Problem}
 Most location detection systems are currently not affordable by the community, and there are not many applications available on the market, which make use of the advantage of person localisation. Therefore the first problem to solve is to find hardware that is affordable and easy to handle. The technology must also have an appropriate accuracy for applications that are utilising the location detection.

\section{Delimitation}
 There are several ways of performing location detection, different hardware devices, and technologies. I limit myself to only one hardware device and one technology, but will design the project such that it should be possible to use other hardware as well. The project will provide a 2 dimensional location detection, with 3 dimensional location detection being outside the scope of the project.

\section{Goal}
 This project will provide a basis of tools which provide an interface to the person location system tracker to easy build applications that can make use of this information. This project also includes a small application which will make use of this system.
